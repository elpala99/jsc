\documentclass[11pt]{article}
\usepackage[british]{babel}
\usepackage{geometry}
% The amsmath package must be before the unicode-math package.
\usepackage{amsmath}
\usepackage{amsthm}
\usepackage{multicol}
\usepackage{graphicx}
\usepackage{hyperref}

\theoremstyle{definition}
\newtheorem{definition}{Definition}[section]

%%%%%%%%%%%%%%%%%%%%%%%%%%%%%%%%%%%%%%%%%%%%%%%%%%%%%%%%%%%%%%%%%%%%%%%%%%%%%%
% Title

\title{Ordinals in Type Theory Notes}
\author{Juan Sebastián Cárdenas}
\date{\today}

%%%%%%%%%%%%%%%%%%%%%%%%%%%%%%%%%%%%%%%%%%%%%%%%%%%%%%%%%%%%%%%%%%%%%%%%%%%%%%
\begin{document}
\maketitle
\section{Some important definitions}

\begin{definition}
  A element $x$ of a set $A$ ordered by the relationship $\leq$ is called the least element if it occurs that $(\forall y \in A) x \leq y$
\end{definition}

\begin{definition}
  A set is well-ordered by a relationship if the relationship is antisymmetric, transitive, has connexity and, for every non-empty subset there is a least element in the set.
\end{definition}

\begin{definition}
  A set $A$ is transitive if it occurs that if $x \in A$ and $y \in x$ then $y \in A$
\end{definition}

\section{Ordinals}
In literature it can be found several definitions for ordinals and the way to construct them. The one used here is:

\begin{definition}
  A set $\alpha$ is an ordinal number iff
  \begin{itemize}
  \item The set is transitive
  \item The set is well-ordered by $\in_A$  
  \end{itemize}
\end{definition}

\end{document}
